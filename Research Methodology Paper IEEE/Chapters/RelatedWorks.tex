\section{Related Works}

The application of ML techniques in predicting diabetes has gained significant attention in recent years. Various studies have explored the efficacy of different models and methodologies, contributing valuable insights into this field. Below are summaries some notable studies that have focused on diabetes prediction using ML:
\subsection{Machine Learning and Deep Learning Predictive Models for Type 2 Diabetes}
This study provides a comprehensive review of various ML and DL models applied to predict type 2 diabetes. The review highlights the effectiveness of models such as Bayesian Networks, SVM, and ensemble methods. The authors found that combining multiple models, particularly ensemble techniques, often results in better performance due to their ability to handle the complexity of diabetes data. The review underscores the importance of feature selection and dimensionality reduction in improving model accuracy and efficiency \cite{Ref15}.
%\cite{chaki2022}.
\subsection{Prediction of Diabetes Disease Using an Ensemble of Machine Learning Models}
This research focuses on developing a robust framework for predicting diabetes using an ensemble of ML models. The study utilized the Iraqi Patient Dataset for Diabetes and addressed common challenges such as data imbalance and missing values through advanced preprocessing techniques. By combining multiple models like k-NN, SVM, DT, and RF, the study achieved high accuracy and AUC scores, demonstrating the potential of ensemble methods in enhancing prediction reliability \cite{Ref16}. 
% \cite{ensemble2021}.
\subsection{ML Models for Data-Driven Prediction of Diabetes}
This study evaluated the efficacy of various ML models in predicting diabetes based on lifestyle and demographic data. The models included LR, DT, RF, GBM, and SVM. The authors found that RF and GBM offered the best performance in terms of accuracy and generalizability. The study also emphasized the importance of cross-validation and rigorous data preprocessing to enhance model robustness and prevent overfitting \cite{Ref17}. 
% \cite{systematic2021}.

\subsection{Real-Time Data Integration and AI}
In the realm of type 1 diabetes management, a study integrated continuous glucose monitoring (CGM) data with artificial intelligence to develop real-time decision support systems. These systems use advanced algorithms to detect glucose level trends and predict potential hypo- or hyperglycemic events, thereby providing timely recommendations for insulin adjustments and lifestyle changes. This real-time integration enhances diabetes management by offering personalized and dynamic support \cite{Ref18}.

These studies collectively highlight the advancements and challenges in using ML for diabetes prediction. They demonstrate that ensemble methods, feature selection, and comprehensive data preprocessing are critical to developing effective predictive models. 

